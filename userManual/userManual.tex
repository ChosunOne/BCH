\documentclass[11pt,a4paper]{article}
\setlength{\textwidth}{440pt} 
\setlength{\oddsidemargin}{5pt}
\setlength{\headheight}{0pt}
\setlength{\textheight}{680pt}
\setlength{\topmargin}{0pt}
\setlength{\headsep}{0pt}
\usepackage[utf8]{inputenc}
\usepackage{amsmath}
\usepackage{amsfonts}
\usepackage{amssymb}
\usepackage{enumerate}   
\usepackage{enumitem}   
\usepackage{fancyvrb}


\newcommand{\nL}{\mathcal{L}}
\newcommand{\ee}{\mathrm{e}}
\newcommand{\dd}{\mathrm{d}}
\renewcommand{\AA}{\mathtt{A}}
\newcommand{\BB}{\mathtt{B}}
\newcommand{\CC}{\mathtt{C}}
\newcommand{\ZZ}{\mathtt{Z}}
\newcommand{\QQ}{\mathbb{Q}}


%\title{User manual for {\tt bch}, a program for the fast computation of the Baker--Campbell--Hausdorff and similar series}
%
%\author{Harald Hofst\"{a}tter}
%
\begin{document}

\begin{center}
    {\LARGE User manual for {\tt bch}, a program for the fast computation of the\\[2mm]
    Baker--Campbell--Hausdorff and similar series}
\vskip 20pt
{\bf Harald Hofst\"atter}\\ %\footnote{any footnote here}}\\
{\small\it Reitschachersiedlung 4/6, 7100 Neusiedl am See, Austria}\\
{\tt hofi@harald-hofstaetter.at}\\ %(optional)
\end{center}
\vskip 20pt



\centerline{\bf Abstract}
\noindent
This manual describes \verb|bch|, 
an efficient  program written in the C programming language for the fast computation
of the Baker--Campbell--Hausdorff (BCH) and similar Lie series.
The Lie series can be represented in the Lyndon basis, in the
classical Hall basis, or in the right-normed basis of 
E.S.~Chibrikov.  In the Lyndon basis,
which proves to be particularly efficient for this purpose,
the computation of 111\,013 coefficients for the BCH series up to terms of degree 20
takes less than half a second on an ordinary personal computer and requires negligible 11\,MB of memory.
Up to terms of degree 30, which is the maximum degree the program can handle, 
the computation of 74\,248\,451 coefficients takes 55 hours but still requires only a modest 5.5\,GB of  memory.


\vskip 30pt

%\maketitle
\section{Introduction}
We consider the element 
\begin{equation}\label{eq:BCH_element}
H = \log(\ee^{\AA}\ee^{\BB})
= \sum_{k=1}^\infty\frac{(-1)^{k+1}}{k}\big(\ee^{\AA}\ee^{\BB}-1\big)^k
= \sum_{k=1}^\infty\frac{(-1)^{k+1}}{k}\bigg(\sum_{i+j>0}\frac{1}{i!j!}\AA^i\BB^j\bigg)^k %=\sum_{w\in\{\AA,\BB\}^*} g_w w
\end{equation}
in the ring $\QQ\langle\langle\AA,\BB\rangle\rangle$  of formal power series 
in the non-commuting variables $\AA$ and $\BB$ with rational coefficients.
This element $H$ is commonly called the Baker--Campbell--Hausdorff (BCH) series. 
A classical result known as the
 Baker--Campbell--Hausdorff theorem (see, e.g.,~\cite{HHproof}) states that 
$H$
is a Lie series, which means that the 
homogeneous components $H_n$ of degree $n=1,2,\dots$ in
\begin{equation}
H=\sum_{n=1}^\infty H_n, %,\quad\mbox{where}\quad X_1=\AA+\BB,\ X_2=\frac{1}{2}[\AA,\BB],\ 
%X_3=\frac{1}{12}[\AA,[\AA,\BB]]+\frac{1}{12}[[\AA,\BB],\BB], \dots
\end{equation}
can be written as  linear combinations of 
$\AA$ and $\BB$ and (possibly nested) 
 commutator terms in $\AA$ and $\BB$, i.e., they
 are 
%Lie polynomials in the 
elements of the
free Lie algebra $\nL_\QQ(\AA,\BB)$ %\subset\QQ\langle\langle\AA,\BB\rangle\rangle$
generated by $\AA$ and $\BB$.  
%Indeed, up to terms of degree 5 we have
%\begin{align*}
%H=&\quad\AA+\BB
%+\frac{1}{2}[\AA,\BB]
%+\frac{1}{12}[A,[A,B]]+\frac{1}{12}[[A,B],B]
%+\frac{1}{24}[A,[[A,B],B]]\\
%&-\frac{1}{720}[A,[A,[A,[A,B]]]]+\frac{1}{180}[A,[A,[[A,B],B]]]+\frac{1}{360}[[A,[A,B]],[A,B]]\\
%&+\frac{1}{180}[A,[[[A,B],B],B]]+\frac{1}{120}[[A,B],[[A,B],B]]-\frac{1}{720}[[[[A,B],B],B],B]+\dots
%\end{align*}

The program \verb|bch| computes the terms of the Lie series $H$ 
up to terms of a given maximal degree $N$, where $H$ (or more precisely, each
component $H_n$, $n=1,\dots,N$) is optionally represented in the Lyndon basis, the right-normed
basis of E.S.~Chibrikov \cite{Chibrikov}, or the classical Hall basis. Furthermore, the program can straightforwardly
be extended to compute the representation in an arbitrary generalized Hall basis.

In addition to $H=\log(\ee^\AA\ee^\BB)$, the program  \verb|bch| can  compute  the Lie series  also for expressions
like the symmetric BCH series
$\log(\ee^{\frac{1}{2}\AA}\ee^\BB\ee^{\frac{1}{2}\AA})$,
the BCH series with 3 generators
$\log(\ee^\AA\ee^\BB\ee^\CC)$,
or the more complex expression 
$\log(\ee^{\frac{1}{6}\BB}\ee^{\frac{1}{2}\AA}
\ee^{\frac{2}{3}\BB+\frac{1}{72}[B,[A,B]}\ee^{\frac{1}{2}\AA}\ee^{\frac{1}{6}\BB})$
with a commutator in an exponential.
The program can straightforwardly be extended to handle arbitrary expressions of
the form
\begin{equation}\label{eq:expression}
X = \log(\ee^{\Phi_s}\cdots\ee^{\Phi_1}),
\end{equation}
 where the $\Phi_i$ are Lie polynomials
%(i.e., linear combinations of commutators) 
with rational coefficients in two or more non-commuting variables.




\section{Implementation}\label{Sec:Imnplementation}
Here we only describe the main ideas on which the implementation of \verb|bch| is based, details will be available in \cite{HHfast}.

The program \verb|bch| is implemented in the  C programming language according to the
C99 (or later) standard. It uses  \verb|khash.h| from the \verb|klib| library \cite{klib}, which is a very efficient implementation of hash tables.
Apart from \verb|khash.h| it is self-contained, i.e., independent of external libraries except the standard C library. 

In particular, the program  uses no library for multi-precision integer or rational arithmetic.
Instead all calculations are carried out in  pure integer arithmetic using the 
128 bit integer type \verb|__int128_t|, which is available for most current C compilers on 
current hardware. With  this integer type computations of the BCH series up to degree
$N=30$ are possible.
To avoid calculations with rational numbers,
the program first determines a common denominator for all rational numbers that can 
occur during the computation (see \cite{HHdenom}, \cite{HHSmallestDenom}), 
then all calculations are reorganized in such a way
that they can actually be carried out in pure integer arithmetic.


In a  preliminary step,
the program  computes the coefficients of all Lyndon words up to degree 
$N$ 
 in the power series of the given expression $X$ of the form (\ref{eq:expression})
using an adaptation of the algorithm from \cite{HHetalWordAlg}. In the special
case $X=\log(\ee^\AA\ee^\AA)$ it alternatively uses the algorithm from 
the appendix of \cite{HHSmallestDenom} and exploits the symmetry that
the coefficient of the word $\AA^{q_1}\BB^{q_2}\cdots(\AA\lor\BB)^{q_m}$ is
invariant under permutations of the exponents $q_1,\dots,q_m$, see \cite{G}.

Next, the program transforms the coefficients of the Lyndon words into coefficients
of Lie basis elements in the Lie series.
For the Lyndon basis and the right-normed Chibrikov basis this is done by solving a linear system whose matrix consists of coefficients of Lyndon words in basis elements.

In the Lyndon basis case this matrix with integer entries is triangular with all diagonal
elements equal 1, which is a consequence of \cite[Theorem~5.1]{Reutenauer}. Thus, this
integer system is readily solved in integer arithmetic with no further denominators introduced.
Here the performance bottleneck is actually the computation of the coefficient matrix,
for which a new efficient algorithm has been developed, for details see \cite{HHfast}. 

In the right-normed Chibrikov basis case, the coefficient matrix is no longer
triangular, but it still has determinant $\pm 1$, and its leading principal
minors are all equal $\pm 1$. It follows that the matrix has a $LU$-decomposition
with triangular matrices $L$, $U$ whose entries are all integers and whose diagonal
elements are equal $\pm 1$. This $LU$-decomposition can be computed 
in pure integer arithmetic using Gaussian elimination without any row or column permutations.
Again solving such a linear system  introduces no further denominators.

In the Hall basis case, the program first computes the representation of
the Lie series in the Lyndon basis, and then uses a standard rewriting algorithm
(see \cite[Section~4.2]{Reutenauer})
applied to the Lyndon basis elements to obtain a representation in the Hall basis.
Because this rewriting algorithm is carried out in integer arithmetic, again no further
denominators are introduced.

The special case $X=\log(\ee^\AA\ee^\BB)$ allows a significant optimization in 
the Lyndon basis or right-normed Chibrikov basis case, if $N$ is even: 
Here the Lie series is computed using the above method only up to terms
of degree $N-1$. The terms of degree $N$ can then efficiently be computed using a
formula due to E. Eriksen, see \cite[Section~III.A]{Eriksen}.\footnote{Note that
this formula is applicable only if $N$ is even, so that, unfortunately, this idea cannot be 
applied recursively.}


As a final remark on the implementation we note that the computation can
be organized in such a way that finely homogeneous components 
$H_{(a,b)}$ of $H$ are computed completely independently of each other.
Here a finely homogeneous component $H_{(a,b)}$ of multi-degree $(a,b)$
is of degree $a$ in $\AA$ and of degree $b$ in $\BB$ such that
$$H_n = \sum_{a+b=n}H_{(a,b)}.$$
Thus, the computations of these components can be done in parallel, giving a
simple but effective parallelization strategy, which 
has been implemented using OpenMP.


\section{Installation}
The source code  can be downloaded from the GitHub repository
\begin{quote}
{\tt https://github.com/HaraldHofstaetter/BCH}
\end{quote}
On a Unix-like system with \verb|gcc| compiler available just type
\begin{quote}
\begin{BVerbatim}
$ make
\end{BVerbatim} 
\end{quote}
in a directory containing the source code, which causes 
the shared library \verb|libbch.so| and the executable \verb|bch|
to be created.
To use a different compiler, the \verb|Makefile| has to be adapted accordingly.


\section{Usage of the {\tt bch} program}\label{Sec:usage_bch}
The program invoked without any arguments produces the following output:

\medskip

{\small\begin{BVerbatim}
$ ./bch
+1/1*A+1/1*B+1/2*[A,B]+1/12*[A,[A,B]]+1/12*[[A,B],B]+1/24*[A,[[A,B],B]]-1/720*[A
,[A,[A,[A,B]]]]+1/180*[A,[A,[[A,B],B]]]+1/360*[[A,[A,B]],[A,B]]+1/180*[A,[[[A,B]
,B],B]]+1/120*[[A,B],[[A,B],B]]-1/720*[[[[A,B],B],B],B]
\end{BVerbatim}
}

\medskip

%\end{quote}
\noindent This is the BCH series (i.e., the Lie series for $\log(\ee^{\AA}\ee^{\BB})$)
up to terms of degree $N=5$ represented in the Lyndon basis.

Arguments have the form
\begin{center}
{\tt\em parameter=value}
\end{center}
Such arguments may be combined arbitrarily. The following parameters are available 
(with default values in square brackets):

\subsection*{{\tt N[=5]}}
The maximal degree up to which the Lie series is computed.

\medskip

{\small\begin{BVerbatim}
$ ./bch N=8
+1/1*A+1/1*B+1/2*[A,B]+1/12*[A,[A,B]]+1/12*[[A,B],B]+1/24*[A,[[A,B],B]]-1/720*[A
,[A,[A,[A,B]]]]+1/180*[A,[A,[[A,B],B]]]+1/360*[[A,[A,B]],[A,B]]+1/180*[A,[[[A,B]
,B],B]]+1/120*[[A,B],[[A,B],B]]-1/720*[[[[A,B],B],B],B]-1/1440*[A,[A,[A,[[A,B],B
]]]]+1/720*[A,[[A,[A,B]],[A,B]]]+1/360*[A,[A,[[[A,B],B],B]]]+1/240*[A,[[A,B],[[A
,B],B]]]-1/1440*[A,[[[[A,B],B],B],B]]+1/30240*[A,[A,[A,[A,[A,[A,B]]]]]]-1/5040*[
A,[A,[A,[A,[[A,B],B]]]]]+1/10080*[A,[A,[[A,[A,B]],[A,B]]]]+1/3780*[A,[A,[A,[[[A,
B],B],B]]]]+1/10080*[[A,[A,[A,B]]],[A,[A,B]]]+1/1680*[A,[A,[[A,B],[[A,B],B]]]]+1
/1260*[A,[[A,[[A,B],B]],[A,B]]]+1/3780*[A,[A,[[[[A,B],B],B],B]]]+1/2016*[[A,[A,B
]],[A,[[A,B],B]]]-1/5040*[[[A,[A,B]],[A,B]],[A,B]]+13/15120*[A,[[A,B],[[[A,B],B]
,B]]]+1/10080*[[A,[[A,B],B]],[[A,B],B]]-1/1512*[[A,[[[A,B],B],B]],[A,B]]-1/5040*
[A,[[[[[A,B],B],B],B],B]]+1/1260*[[A,B],[[A,B],[[A,B],B]]]-1/2016*[[A,B],[[[[A,B
],B],B],B]]-1/5040*[[[A,B],B],[[[A,B],B],B]]+1/30240*[[[[[[A,B],B],B],B],B],B]+1
/60480*[A,[A,[A,[A,[A,[[A,B],B]]]]]]-1/15120*[A,[A,[A,[[A,[A,B]],[A,B]]]]]-1/100
80*[A,[A,[A,[A,[[[A,B],B],B]]]]]+1/20160*[A,[[A,[A,[A,B]]],[A,[A,B]]]]-1/20160*[
A,[A,[A,[[A,B],[[A,B],B]]]]]+1/2520*[A,[A,[[A,[[A,B],B]],[A,B]]]]+23/120960*[A,[
A,[A,[[[[A,B],B],B],B]]]]+1/4032*[A,[[A,[A,B]],[A,[[A,B],B]]]]-1/10080*[A,[[[A,[
A,B]],[A,B]],[A,B]]]+13/30240*[A,[A,[[A,B],[[[A,B],B],B]]]]+1/20160*[A,[[A,[[A,B
],B]],[[A,B],B]]]-1/3024*[A,[[A,[[[A,B],B],B]],[A,B]]]-1/10080*[A,[A,[[[[[A,B],B
],B],B],B]]]+1/2520*[A,[[A,B],[[A,B],[[A,B],B]]]]-1/4032*[A,[[A,B],[[[[A,B],B],B
],B]]]-1/10080*[A,[[[A,B],B],[[[A,B],B],B]]]+1/60480*[A,[[[[[[A,B],B],B],B],B],B
]]
\end{BVerbatim}
}


\subsection*{\tt basis[=0]}
With default value \verb|basis=0| the result is represented in the 
Lyndon basis;   with  value \verb|basis=1| in the right-normed Chibrikov basis:

\medskip

{\small\begin{BVerbatim}
$ ./bch basis=1
+1/1*A+1/1*B-1/2*[B,A]-1/12*[A,[B,A]]+1/12*[B,[B,A]]+1/24*[B,[A,[B,A]]]+1/720*[A
,[A,[A,[B,A]]]]-1/360*[B,[A,[A,[B,A]]]]+1/120*[A,[B,[A,[B,A]]]]-1/120*[B,[B,[A,[
B,A]]]]+1/360*[A,[B,[B,[B,A]]]]-1/720*[B,[B,[B,[B,A]]]]
\end{BVerbatim}
}

\medskip

\noindent With value \verb|basis=2| the result is represented in the classical Hall basis:

\medskip

{\small\begin{BVerbatim}
$ ./bch basis=2
+1/1*A+1/1*B-1/2*[B,A]+1/12*[[B,A],A]-1/12*[[B,A],B]+1/24*[[[B,A],A],B]-1/720*[[
[[B,A],A],A],A]-1/180*[[[[B,A],A],A],B]+1/180*[[[[B,A],A],B],B]+1/720*[[[[B,A],B
],B],B]-1/120*[[[B,A],A],[B,A]]-1/360*[[[B,A],B],[B,A]]
\end{BVerbatim}
}

\medskip

%\noindent [TODO:] other bases...

\subsection*{\tt generators[=\tt ABC\dots]}
This parameter enables alternative names for the generators:
\medskip

{\small\begin{BVerbatim}
$ ./bch generators=xy
+1/1*x+1/1*y+1/2*[x,y]+1/12*[x,[x,y]]+1/12*[[x,y],y]+1/24*[x,[[x,y],y]]-1/720*[x
,[x,[x,[x,y]]]]+1/180*[x,[x,[[x,y],y]]]+1/360*[[x,[x,y]],[x,y]]+1/180*[x,[[[x,y]
,y],y]]+1/120*[[x,y],[[x,y],y]]-1/720*[[[[x,y],y],y],y]
\end{BVerbatim}
}

\medskip

\subsection*{\tt expression[=0]}
\begin{itemize}[leftmargin=*]
\item {\tt expression=0} -- compute Lie series for $\log(\ee^\AA\ee^\BB)$, the classical BCH formula.
  \item {\tt expression=1} -- compute Lie series for $\log(\ee^{\frac{1}{2}\AA}\ee^\BB\ee^{\frac{1}{2}\AA})$, the symmetric BCH formula:

{\small\begin{BVerbatim}
$./bch expression=1
+1/1*A+1/1*B-1/24*[A,[A,B]]+1/12*[[A,B],B]+7/5760*[A,[A,[A,[A,B]]]]-7/1440*[A,[A
,[[A,B],B]]]+1/360*[[A,[A,B]],[A,B]]+1/180*[A,[[[A,B],B],B]]+1/120*[[A,B],[[A,B]
,B]]-1/720*[[[[A,B],B],B],B]
\end{BVerbatim}
}

\item {\tt expression=2} -- compute Lie series for $\log(\ee^\AA\ee^\BB\ee^\AA)$, another symmetric version of the BCH formula:

{\small\begin{BVerbatim}
$./bch expression=2
+2/1*A+1/1*B-1/6*[A,[A,B]]+1/6*[[A,B],B]+7/360*[A,[A,[A,[A,B]]]]-7/180*[A,[A,[[A
,B],B]]]+1/45*[[A,[A,B]],[A,B]]+1/45*[A,[[[A,B],B],B]]+1/30*[[A,B],[[A,B],B]]-1/
360*[[[[A,B],B],B],B]
\end{BVerbatim}
}

\item {\tt expression=3} -- compute Lie series for $\log(\ee^\AA\ee^\BB\ee^\CC)$, a BCH
formula with three different exponentials: 

{\small\begin{BVerbatim}
$./bch expression=3 N=4
+1/1*A+1/1*B+1/1*C+1/2*[A,B]+1/2*[A,C]+1/2*[B,C]+1/12*[A,[A,B]]+1/12*[A,[A,C]]+1
/12*[[A,B],B]+1/3*[A,[B,C]]+1/6*[[A,C],B]+1/12*[[A,C],C]+1/12*[B,[B,C]]+1/12*[[B
,C],C]+1/24*[A,[[A,B],B]]+1/12*[A,[A,[B,C]]]+1/12*[A,[[A,C],B]]+1/24*[A,[[A,C],C
]]+1/12*[A,[B,[B,C]]]+1/12*[[A,[B,C]],B]+1/12*[A,[[B,C],C]]+1/12*[[A,C],[B,C]]+1
/24*[B,[[B,C],C]]
\end{BVerbatim}
}

\item {\tt expression=4} -- compute Lie series for $\log(\ee^\AA\ee^\BB\ee^{-\AA}\ee^{-\BB})$:
    
{\small\begin{BVerbatim}
$./bch expression=4 
+1/1*[A,B]+1/2*[A,[A,B]]-1/2*[[A,B],B]+1/6*[A,[A,[A,B]]]-1/4*[A,[[A,B],B]]+1/6*[
[[A,B],B],B]+1/24*[A,[A,[A,[A,B]]]]-1/12*[A,[A,[[A,B],B]]]+1/12*[[A,[A,B]],[A,B]
]+1/12*[A,[[[A,B],B],B]]-1/24*[[[[A,B],B],B],B]
\end{BVerbatim}
}

\item {\tt expression=5} -- compute Lie series for $\log(\ee^{\frac{1}{6}\BB}\ee^{\frac{1}{2}\AA}
\ee^{\frac{2}{3}\BB+\frac{1}{72}[B,[A,B]}\ee^{\frac{1}{2}\AA}\ee^{\frac{1}{6}\BB})$,
a more complicated version of the BCH formula, which is constructed in such a way that 
besides $\AA+\BB$ all terms have degree $\geq 5$:

{\small\begin{BVerbatim}
$./bch expression=5
+1/1*A+1/1*B+1/2880*[A,[A,[A,[A,B]]]]-7/8640*[A,[A,[[A,B],B]]]+1/2160*[[A,[A,B]]
,[A,B]]+7/12960*[A,[[[A,B],B],B]]+1/4320*[[A,B],[[A,B],B]]-41/155520*[[[[A,B],B]
,B],B]
\end{BVerbatim}
}
\end{itemize}
The program can easily be extended to handle additional expressions
of the form (\ref{eq:expression}), see Subsection~\ref{SubSec:UserDefExpr} below.

\subsection*{\tt table\_output[=0 {\em or} =1 {\em depending on size of result}]}
With \verb|table_output=0| the result is displayed as a linear combination of commutators. 
This is the default if the result consists of less than 200 terms.
With
\verb|table_output=1|, on the other hand, the result is displayed in tabular form.
Each row of the table corresponds to a basis element of the respective Lie basis,
%chosen by the \verb|basis| parameter, 
where by default, for the Lyndon or the Hall
basis (\verb|basis=0| or \verb|basis=2|),
the index, the indices of the left and
the right factors, and the coefficient of the basis element is displayed. 
This information uniquely determines the basis element, which follows from the
fact that 
for Hall and Lyndon bases the factors $h', h''$ of basis elements $h=[h',h'']$ of degree $\geq 2$ are
basis elements themselves.
\begin{quote} %
{\small\begin{BVerbatim}
$ ./bch table_output=1
0       1       0       0       1/1
1       1       1       0       1/1
2       2       0       1       1/2
3       3       0       2       1/12
4       3       2       1       1/12
5       4       0       3       0/1
6       4       0       4       1/24
7       4       4       1       0/1
8       5       0       5       -1/720
9       5       0       6       1/180
10      5       3       2       1/360
11      5       0       7       1/180
12      5       2       4       1/120
13      5       7       1       -1/720
\end{BVerbatim}
}\end{quote}
For the right-normed basis (\verb|basis=1|) the factors of
the basis element are not necessarily
basis elements themselves, so that instead of these factors the foliage 
of the basis element (i.e., the commutator written without commas and brackets)
is displayed, which uniquely determines the basis element  in this case.
\begin{quote} %
{\small\begin{BVerbatim}
$ ./bch basis=1 table_output=1
0       1       A       1/1
1       1       B       1/1
2       2       BA      -1/2
3       3       ABA     -1/12
4       3       BBA     1/12
5       4       AABA    0/1
6       4       BABA    1/24
7       4       BBBA    0/1
8       5       AAABA   1/720
9       5       BAABA   -1/360
10      5       ABABA   1/120
11      5       BBABA   -1/120
12      5       ABBBA   1/360
13      5       BBBBA   -1/720
\end{BVerbatim}
}\end{quote}
The displayed items can be customized by specifying the following parameters
which should be self-explanatory:
\begin{itemize}
\item {\tt print\_index[=1]}
\item {\tt print\_degree[=1]}
\item {\tt print\_multi\_degree[=0]}
\item {\tt print\_factors[=1 {\em or} =0 {\em depending on} basis]}
\item {\tt print\_foliage[=0 {\em or} =1 {\em depending on} basis]}
\item {\tt print\_basis\_element[=0]}
\item {\tt print\_coefficient[=1]}
\end{itemize}
\begin{quote} %
{\small\begin{BVerbatim}
$ ./bch table_output=1 print_basis_element=1 print_multi_degree=1
0       1       (1,0)   0       0       A       1/1
1       1       (0,1)   1       0       B       1/1
2       2       (1,1)   0       1       [A,B]   1/2
3       3       (2,1)   0       2       [A,[A,B]]       1/12
4       3       (1,2)   2       1       [[A,B],B]       1/12
5       4       (3,1)   0       3       [A,[A,[A,B]]]   0/1
6       4       (2,2)   0       4       [A,[[A,B],B]]   1/24
7       4       (1,3)   4       1       [[[A,B],B],B]   0/1     
8       5       (4,1)   0       5       [A,[A,[A,[A,B]]]]       -1/720
9       5       (3,2)   0       6       [A,[A,[[A,B],B]]]       1/180
10      5       (3,2)   3       2       [[A,[A,B]],[A,B]]       1/360
11      5       (2,3)   0       7       [A,[[[A,B],B],B]]       1/180
12      5       (2,3)   2       4       [[A,B],[[A,B],B]]       1/120
13      5       (1,4)   7       1       [[[[A,B],B],B],B]       -1/720
\end{BVerbatim}
}\end{quote}
Note that if the parameter  \verb|verbosity_level| 
is set to a value $\geq 1$ (see below), then  a header line for the table 
is printed.

\subsection*{\tt verbosity\_level[=0]}
With \verb|verbosity_level| set to a value $\geq 1$ some useful information about
the performance of the computation is printed, for example the common denominator 
%used for the computation 
and timings of the main steps of the computation as described in Section~\ref{Sec:Imnplementation}. Furthermore a table showing the number of the
Lie basis elements and the number of nonzero coefficients of the Lie series is printed.
For easy identification, each line of this output has the character \verb|#| at the leading position.

With \verb|verbosity_level| set to a value $\geq 2$ this  output is not buffered but flushed immediately. 
This is only recommended if long running times of the program are expected.

\begin{quote} %
{\small\begin{verbatim}
$ ./bch N=20 verbosity_level=1 | head -n 40
#number of Lyndon words of length<=20 over set of 2 letters: 111013
#initialize Lyndon words: time=0.0344573 sec
#expression=log(exp(A)*exp(B))
#denominator=102181884343418880000
#compute Goldberg coefficients: time=0.0137833 sec
#compute coefficients of Lyndon words: time=0.0226991 sec
#convert to Lie series: time=0.158603 sec
#compute terms of degree 20: time=0.000395872 sec
#total time=0.216212 sec
# degree         dim    #nonzero   dim(cum.)   #nz(cum.)
#      1           2           2           2           2
#      2           1           1           3           3
#      3           2           2           5           5
#      4           3           1           8           6
#      5           6           6          14          12
#      6           9           5          23          17
#      7          18          18          41          35
#      8          30          17          71          52
#      9          56          55         127         107
#     10          99          55         226         162
#     11         186         186         412         348
#     12         335         185         747         533
#     13         630         630        1377        1163
#     14        1161         629        2538        1792
#     15        2182        2181        4720        3973
#     16        4080        2181        8800        6154
#     17        7710        7710       16510       13864
#     18       14532        7709       31042       21573
#     19       27594       27594       58636       49167
#     20       52377       27593      111013       76760
#
# i     |i|     i'      i"      coefficient
0       1       0       0       1/1
1       1       1       0       1/1
2       2       0       1       1/2
3       3       0       2       1/12
4       3       2       1       1/12
5       4       0       3       0/1
6       4       0       4       1/24
7       4       4       1       0/1
\end{verbatim}
}\end{quote}

\section{Performance of the {\tt bch} program}
We document the performance of typical runs of the \verb|bch|
program on an ordinary personal computer.
More specifically, the computer system was a
a 3.0\,GHz Intel Core i5-2320 processor with 4 CPU cores and
8\,GB of RAM running an Ubuntu Linux operating system. The 
C compiler was \verb|gcc| version 9.3.0.
The memory usage was measured with \verb|/usr/bin/time|.
For the test runs of the program we used different 
maximum degrees $N$ and different bases. The results
are as follows: (Here, dimension is the number of basis elements of degree $\leq N$
and \#nonzero is the number of coefficients $\neq 0$.)

\subsubsection*{Lyndon basis:}

\nopagebreak

\begin{center}
\begin{tabular}{rrrrr}
\hline
$N$ & dimension & \#nonzero & time  & memory \\
\hline
18 &  31042 & 21573 &  $0.04$\,sec & 4628\,kB\\
20 & 111013 & 76760 & $0.22$\,sec  & 10932\,kB \\
22 & 401428 & 276474 & $2.13$\,sec & 33008\,kB \\
24 & 1465020 & 1005917 & $54.43$\,sec  & 115\,MB\\
26 & 5387991 & 3690268 & 16\,min 16\,sec  &  427\,MB\\
28 & 19945394 & 13632278 & 3\,h 57\,min & 1.4\,GB\\
30 & 74248451 & 50657857 & 54\,h 46\,min & 5.5\,GB\\
\hline
\end{tabular}
\end{center}

\subsubsection*{Classical Hall basis:}
\begin{center}
\begin{tabular}{rrrrr}
\hline
$N$ & dimension & \#nonzero & time  & memory \\
\hline
18 & 31042 & 30477   & $0.35$\,sec & 81248\,kB\\
20 & 111013 & 109697 & $2.12$\,sec & 428\,MB \\
22 & 401428 & 398313 & 20\,min 53\,sec & 2.6\,GB\\
\hline
\end{tabular}
\end{center}

\subsubsection*{Right-normed Chibrikov basis:}
\begin{center}
\begin{tabular}{rrrrr}
\hline
$N$ & dimension & \#nonzero & time  & memory \\
\hline
18 & 31042  & 21561 & $0.76$\,sec & 58300\,kB \\
20 & 111013 & 76748 & $21.52$\,sec & 637\,MB \\
22 & 401428 & 276463 & 21\,min 53\,sec & 6.9\,GB\\
\hline
\end{tabular}
\end{center}

For the Lyndon basis, the maximum degree $N=30$ that the program
can in principle handle could actually be achieved.
For the other bases, $N=22$ was the limit due to memory constraints
(the computer system had 8\,GB of RAM).

It is natural to compare the performance of our program with that
of other implementations of algorithms for computing the BCH series
up to terms of high degree.
Ref.~\cite{CasasMurua} is the only recently published report 
of such an implementation
that we know of. 
Results obtained with this implementation are available at 
\begin{quote}
\verb|http://www.ehu.eus/ccwmuura/bch.html|.
\end{quote}    
It should be acknowledged that these results served us well in verifying the computations of our \verb|bch| program.

For the implementation of \cite{CasasMurua} in {\sc Mathematica}
a completely different and  more sophisticated 
approach than ours was used. This approach, which is
based on a Lie algebraic structure of labeled rooted trees,
has the  obvious disadvantage  that
it is not easily adapted to arbitrary expressions 
of the form (\ref{eq:expression}). Thus, only the classical and the symmetric BCH series are handled in \cite{CasasMurua}.
%Furthermore, our simpler approach is arguably better suited for massive optimizations.

In \cite{CasasMurua} it is reported  that the computation of the BCH series
up to terms of degree $N=20$ in the classical Hall basis required  less than
15\,min CPU time and 1.5\,GB of memory 
on a 2.4\,GHz Intel Core 2 Duo processor with 2\,GB of RAM.
For the computation in the Lyndon basis it required 3.6\,GB of memory, but no
specific information about the CPU time is given, only that
it required more time than the computation in the classical Hall basis.
This has to be compared with  our implementation, which  requires
only 0.22\,sec CPU time and 11\,MB of RAM for the Lyndon basis, 
and 2.12\,sec CPU time and 428\,MB of RAM for the classical Hall basis.
These numbers speak for themselves, even if  direct comparisons 
of different implementations in different programming languages that run on different hardware 
must be interpreted with caution.

\section{Usage of the {\tt libbch} library}
The \verb|main()| function 
of  the executable \verb|bch|, which is
defined in the source file \verb|bch.c|, essentially does nothing else
than processing the command line arguments and then calling 
the appropriate functions from the \verb|libbch| library, which
perform the actual computations.
In the same way, these functions can also be called from a user-defined program,
which has to be linked to the 
shared library \verb|libbch.so| for this purpose.
Each such program should include the header file
\verb|bch.h|, which contains declarations 
for the functions and data structures provided by the library.
The source file \verb|bch.c| for the executable \verb|bch| can serve as a model for
the usage of the  \verb|libbch| library.

In the following we give an 
excerpt from the header file \verb|bch.h|
covering  essential declarations together with some
necessary explanations. 
It has been tried to use names for the functions and their arguments
that are as self-explanatory as possible, and that are consistent with
the names of the corresponding parameters for the  \verb|bch| program
as described in Section~\ref{Sec:usage_bch}.

\subsection{128 bit integer type}
Essentially all computations of the \verb|libbch| library are carried out in pure integer arithmetic, a
large part of them in 128 bit integer arithmetic (the other part can be done
using integers with fewer bits). In \verb|bch.h| 
the 128 bit integer type  \verb|INTEGER| is defined
as an alias for the built-in type \verb|__int128_t|:
\begin{verbatim}
typedef __int128_t INTEGER; 
\end{verbatim}
Although such 128 bit integers are available for most modern
C compilers, the C %programming language and its 
standard library does not
provide functions for converting such integers to strings, or print them 
on standard output. The \verb|libbch| library provides such functions and also functions 
for converting and printing rational numbers, which are 
specified by their numerators and denominators of type \verb|INTEGER|.
Note that the rational numbers are reduced to lowest terms first.
\begin{verbatim}
int str_INTEGER(char *str, INTEGER x);
int str_RATIONAL(char *str, INTEGER num, INTEGER den);
void print_INTEGER(INTEGER x);
void print_RATIONAL(INTEGER num, INTEGER den);
\end{verbatim}
The functions \verb|str_INTEGER| and \verb|str_RATIONAL| write
to a character string \verb|str| which has to be long enough 
to contain the output including a terminating \verb|'\0'|.
They return the number of characters written (excluding the terminating \verb|'\0'|).
If \verb|str| is \verb|NULL|, then these functions return the number
of characters that would have been written in case of a proper output
string \verb|str|.


\subsection{Lie series}
The following two functions compute respectively the classical BCH series
(i.e., the Lie series for $\log(\ee^{\AA}\ee^{\BB})$) and
the symmetric BCH series (i.e., the Lie series for 
$\log(\ee^{\frac{1}{2}\AA}\ee^{\BB}\ee^{\frac{1}{2}\AA}))$:
\begin{verbatim}
lie_series_t* BCH(int maximum_degree, int basis);
lie_series_t* symBCH(int maximum_degree, int basis);
\end{verbatim}
Here the argument \verb|maximum_degree| specifies the maximum degree
up to which terms of the Lie series shall be computed. The argument
\verb|basis| specifies the basis in which the resulting Lie series
shall be represented, where the values 0,1,2 respectively correspond to the
Lyndon basis, the right-normed Chibrikov basis, and the classical Hall basis.
These functions return a pointer to a structure of type \verb|lie_series_t| which 
contains the resulting Lie series. 
This structure should be considered as an opaque data type for 
which access to its data is provided by data access functions (see below).



\begin{verbatim}
void set_verbosity_level(int verbosity_level);
\end{verbatim}
If the verbosity level is set to a value $\geq 1$ {\em before} calling
the functions \verb|BCH| or \verb|symBCH|, then these function print
some useful information about the performance of the computation to the standard output.
%\begin{verbatim}
%int get_verbosity_level(void);
%\end{verbatim}

\subsection{Data access functions}
\begin{verbatim}
int dimension(lie_series_t *LS);
int maximum_degree(lie_series_t *LS);
int number_of_generators(lie_series_t *LS);
INTEGER denominator(lie_series_t *LS);
INTEGER numerator_of_coefficient(lie_series_t *LS,  int i);
int degree(lie_series_t *LS, int i);
int degree_of_generator(lie_series_t *LS, int i, uint8_t g);
int left_factor(lie_series_t *LS, int i);
int right_factor(lie_series_t *LS, int i);
int str_foliage(char *str, lie_series_t *LS, int i, char *generators);
int str_basis_element(char *str, lie_series_t *LS, int i, char *generators);
int str_coefficient(char *str, lie_series_t *LS, int i);
void print_foliage(lie_series_t *LS, int i, char *generators);
void print_basis_element(lie_series_t *LS, int i, char *generators);
void print_coefficient(lie_series_t *LS,  int i);
\end{verbatim}
Here, the argument \verb|LS| is a pointer which was
previously returned by the functions \verb|BCH| or \verb|symBCH|
(or \verb|lie_series|, see below).
The argument \verb|i| is the index of the basis element, for which
the respective information is requested.
The argument \verb|g| in \verb|degree_of_generator| specifies
the (index of the) generator for which the degree in the basis element with
index \verb|i| is requested.
The argument \verb|generators| should be a string like \verb|"ABCDEFG"| containing at 
position 0 the name for generator 0, at position 1 the name for generator 1, etc.

The functions \verb|str_foliage|, \verb|str_basis_element|, and \verb|str_coefficient| write
to a character string \verb|str| which has to be long enough 
to contain the output including a terminating \verb|'\0'|.
They return the number of characters written (excluding the terminating \verb|'\0'|).
If \verb|str| is \verb|NULL|, then these functions return the number
of characters that would have been written in case of a proper output
string \verb|str|.

The functions \verb|left_factor| and \verb|right_factor|
respectively
return the index of the left factor $h'$ and the index of the right
factor $h''$ of the basis element $h=[h',h'']$ with index \verb|i|.
Note that the result is well-defined only if the Lie series \verb|LS| 
was generated with  \verb|basis| equal 0 (Lyndon basis) or 
equal 2 (classical Hall basis),
and if the basis element $h$ with index \verb|i| has degree $\geq 2$.

\subsection{Cleaning up}
\begin{verbatim}
void free_lie_series(lie_series_t *LS);
\end{verbatim}


\subsection{User-defined expressions}\label{SubSec:UserDefExpr}
Besides Lie series for  $\log(\ee^{\AA}\ee^{\BB})$ and
$\log(\ee^{\frac{1}{2}\AA}\ee^{\BB}\ee^{\frac{1}{2}\AA)})$, also 
Lie series for arbitrary expressions of the form (\ref{eq:expression})
can be computed.
Such expressions are represented as  binary expression
trees, for the generation of which  the \verb|libbch| library
provides the following functions:
%\begin{verbatim}
%enum expr_type { UNDEFINED, IDENTITY, GENERATOR, SUM, DIFFERENCE, PRODUCT, 
                 %NEGATION, TERM, EXPONENTIAL, LOGARITHM };
%
%typedef struct expr_t {
    %enum expr_type type;
    %struct expr_t *arg1;
    %struct expr_t *arg2;
    %int num;
    %int den;
%} expr_t;
%\end{verbatim}
\begin{verbatim}
expr_t* identity(void);
expr_t* generator(uint8_t g);
expr_t* sum(expr_t* arg1, expr_t* arg2);
expr_t* difference(expr_t* arg1, expr_t* arg2);
expr_t* product(expr_t* arg1, expr_t* arg2);
expr_t* negation(expr_t* arg);
expr_t* term(int numerator, int denominator, expr_t* arg);
expr_t* exponential(expr_t* arg);
expr_t* logarithm(expr_t* arg);
expr_t* commutator(expr_t* arg1, expr_t* arg2);
\end{verbatim}
These functions return a pointer to a structure of type \verb|expr_t|
which represents a node of the expression tree and contains data,
depending on the type of the node,
like pointers to subexpressions, the index of a generator, or numerator and
denominator of a rational coefficient.
For example, the following code generates the 
expression 
$\log(\ee^{\frac{1}{6}\BB}\ee^{\frac{1}{2}\AA}
\ee^{\frac{2}{3}\BB+\frac{1}{72}[B,[A,B]}\ee^{\frac{1}{2}\AA}\ee^{\frac{1}{6}\BB})$:
%(see also further examples in the source code file \verb|bch.c|
%for the executable \verb|bch|):
\begin{quote} %
{\small\begin{BVerbatim}
expr_t *A = generator(0);
expr_t *B = generator(1);
expr_t* expression =
    logarithm(product(product(product(product(
    exponential(term(1, 6, B)), exponential(term(1, 2, A))),
    exponential(sum(term(2, 3, B), 
                    term(1, 72, commutator(B, commutator(A, B)))))), 
    exponential(term(1, 2, A))), exponential(term(1, 6, B))));
\end{BVerbatim}
}
\end{quote}
For an expression obtained in this way, the following function
computes the corresponding Lie series:\footnote{Of course, it is possible to generate more general expressions 
than those of the form (\ref{eq:expression}). 
For such more general expression, the result of the function {\tt lie\_series} does not necessarily make any sense, 
since it cannot be guaranteed that such expressions can be represented as  Lie series or even
as  formal power series.}
%
%\begin{verbatim}
%int str_expr(char *str, expr_t* ex);
%void print_expr(expr_t* ex);
%\end{verbatim}
%
%\begin{verbatim}
%int phi(INTEGER y[], int m, uint8_t w[], expr_t* ex, INTEGER v[]);
%INTEGER common_denominator(int n, expr_t* ex);
%\end{verbatim}
%
\begin{verbatim}
lie_series_t* lie_series(int number_of_generators, expr_t* expression, 
                         int maximum_degree, int basis);
\end{verbatim}
Regarding arguments and return value, this function is analogous to the functions \verb|BCH| and \verb|symBCH|,
with the exception of the additional arguments \verb|number_of_generators| and \verb|expression|. 
Here, \verb|number_of_generators| is the number of generators occurring in the expression (or more
precisely the highest index plus 1 of all generators occurring in the expression). In the above example
it should have the value 2.

For cleaning up, the following function frees all resources that were allocated for an expression:
\begin{verbatim}
void free_expr(expr_t* expression);
\end{verbatim}
Expressions should be freed in the same order as they were generated, i.e., for the above example:
\begin{quote} %
{\small\begin{BVerbatim}
free_expr(A);
free_expr(B);
free_expr(expression);
\end{BVerbatim}
}
\end{quote}




\begin{thebibliography}{10}\footnotesize

\bibitem{CasasMurua}
F. Casas and A. Murua, An efficient algorithm for computing the Baker--Campbell--Hausdorff series
and some of its applications, {\em J. Math. Phys.} {\bf 50}, 033513 (2009).

\bibitem{Chibrikov}
E.S. Chibrikov, A right normed basis for free Lie algebras and Lyndon--Shirshov
  words, {\em Journal of Algebra} {\bf 302} (2006), 593--612.

%\bibitem{Eichler} M. Eichler, A new proof of the Baker--Campbell-Hausdorff formula,
%{\it J. Math. Soc. Japan}, {\bf 20} (1968), 23--25.

\bibitem{Eriksen} E. Eriksen, Properties of higher-order commutator products and the
    Baker--Campbell--Hausdorff formula, {\em J. Math. Phys.} {\bf 9} (1968), 790--796.

\bibitem{G} K. Goldberg, The formal power series for $\log e^x e^y$, {\it Duke Math. J.} {\bf 23} (1956), 13--21.

\bibitem{HHproof} H. Hofst\"atter, A relatively short self-contained proof of the Baker--Campbell--Hausdorff theorem, {\it Expositiones Mathematicae}, to appear. %online available at {\tt https://doi.org/10.1016/j.exmath.2020.05.003}.


\bibitem{HHdenom} H. Hofst\"atter, Denominators of coefficients of the Baker--Campbell--Hausdorff series, available at {\tt https://arxiv.org/abs/2010.03440}.

\bibitem{HHSmallestDenom} H. Hofst\"atter, Smallest common denominators for the homogeneous components of the Baker--Campbell--Hausdorff series, available at {\tt https://arxiv.org/abs/2012.03818}.

\bibitem{HHfast} H. Hofst\"atter, Fast computation of the Baker--Campbell--Hausdorff and similar series, in preparation.


\bibitem{HHetalWordAlg} H. Hofstätter, W. Auzinger, and O. Koch,   An Algorithm for Computing Coefficients of Words in Expressions Involving Exponentials and Its Application to the Construction of Exponential Integrators, In: M. England, W. Koepf, T. Sadykov, W. Seiler, and E. Vorozhtsov (eds), {\it Computer Algebra in Scientific Computing, CASC 2019}, Lecture Notes in Computer Science {\bf 11661}, Springer, Cham, 2019. 

\bibitem{klib}  Klib, A standalone and lightweight C library, available at {\tt https://github.com/attractivechaos/klib}.

%\bibitem{KellnerSondow} B. C. Kellner and J. Sondow, Power-sum denominators, {\it Amer. Math. %Monthly} {\bf 124} (2017), 695--709.

%\bibitem{Mihet} D. Mihet, Legendre's and Kummer's theorems again, {\it Resonance} {\bf 15}, %(2010),  1111-1121. 

%\bibitem{NewmanThompson} M. Newman and R. C. Thompson, Numerical values of Goldberg's %coefficients in the series for $\log(e^xe^y)$, {\it Math. Comput.} {\bf 48} (1987), 256--271.

%\bibitem{Reinsch} M. W. Reinsch, A simple expression for the terms in the
%  Baker--Campbell--Hausdorff series, {\it J. Math. Phys.} {\bf 41} (2000), 2434-2442.


\bibitem{Reutenauer} C. Reutenauer, Free Lie Algebras, Oxford University Press, Oxford 1993.

%\bibitem{SloaneOEIS} N. J. A. Sloane, ed., The Online Encyclopedia of Integer Sequences, {\tt http://oeis.org}.

%\bibitem{VanBruntVisser} A. Van-Brunt and M. Visser, Simplifying the Reinsch algorithm for the Baker--Campbell--Hausdorff series, {\it J. Math. Phys.} {\bf 57} (2016), 023507.

%\bibitem{A} J. Author, My first math paper, {\it Integers} {\bf 25}, \#A32.
%\bibitem{CD} I. Can and N. Do, Proof of existence, {\it J. Math. Stuff} {\bf 17}, 19-23.

\end{thebibliography} 

\end{document}
